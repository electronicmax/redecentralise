% THIS IS SIGPROC-SP.TEX - VERSION 3.1
% WORKS WITH V3.2SP OF ACM_PROC_ARTICLE-SP.CLS
% APRIL 2009
%
% It is an example file showing how to use the 'acm_proc_article-sp.cls' V3.2SP
% LaTeX2e document class file for Conference Proceedings submissions.
% ----------------------------------------------------------------------------------------------------------------
% This .tex file (and associated .cls V3.2SP) *DOES NOT* produce:
%       1) The Permission Statement
%       2) The Conference (location) Info information
%       3) The Copyright Line with ACM data
%       4) Page numbering
% ---------------------------------------------------------------------------------------------------------------
% It is an example which *does* use the .bib file (from which the .bbl file
% is produced).
% REMEMBER HOWEVER: After having produced the .bbl file,
% and prior to final submission,
% you need to 'insert'  your .bbl file into your source .tex file so as to provide
% ONE 'self-contained' source file.
%
% Questions regarding SIGS should be sent to
% Adrienne Griscti ---> griscti@acm.org
%
% Questions/suggestions regarding the guidelines, .tex and .cls files, etc. to
% Gerald Murray ---> murray@hq.acm.org
%
% For tracking purposes - this is V3.1SP - APRIL 2009

\documentclass{article}

\begin{document}

\title{Re-decentralising the Web}
%\subtitle{[A Proposal]
% \titlenote{A full version of this paper is available as
% \textit{Author's Guide to Preparing ACM SIG Proceedings Using
% \LaTeX$2_\epsilon$\ and BibTeX} at
% \texttt{www.acm.org/eaddress.htm}}}
%
% You need the command \numberofauthors to handle the 'placement
% and alignment' of the authors beneath the title.
%
% For aesthetic reasons, we recommend 'three authors at a time'
% i.e. three 'name/affiliation blocks' be placed beneath the title.
%
% NOTE: You are NOT restricted in how many 'rows' of
% "name/affiliations" may appear. We just ask that you restrict
% the number of 'columns' to three.
%
% Because of the available 'opening page real-estate'
% we ask you to refrain from putting more than six authors
% (two rows with three columns) beneath the article title.
% More than six makes the first-page appear very cluttered indeed.
%
% Use the \alignauthor commands to handle the names
% and affiliations for an 'aesthetic maximum' of six authors.
% Add names, affiliations, addresses for
% the seventh etc. author(s) as the argument for the
% \additionalauthors command.
% These 'additional authors' will be output/set for you
% without further effort on your part as the last section in
% the body of your article BEFORE References or any Appendices.

\numberofauthors{5} %  in this sample file, there are a *total*
% of EIGHT authors. SIX appear on the 'first-page' (for formatting
% reasons) and the remaining two appear in the \additionalauthors section.
%
\author{
% You can go ahead and credit any number of authors here,
% e.g. one 'row of three' or two rows (consisting of one row of three
% and a second row of one, two or three).
%
% The command \alignauthor (no curly braces needed) should
% precede each author name, affiliation/snail-mail address and
% e-mail address. Additionally, tag each line of
% affiliation/address with \affaddr, and tag the
% e-mail address with \email.
%
% 1st. author
% \alignauthor
% Ben Trovato\titlenote{Dr.~Trovato insisted his name be first.}\\
%        \affaddr{Institute for Clarity in Documentation}\\
%        \affaddr{1932 Wallamaloo Lane}\\
%        \affaddr{Wallamaloo, New Zealand}\\
%        \email{trovato@corporation.com}
% % 2nd. author
% \alignauthor
% G.K.M. Tobin\titlenote{The secretary disavows
% any knowledge of this author's actions.}\\
%        \affaddr{Institute for Clarity in Documentation}\\
%        \affaddr{P.O. Box 1212}\\
%        \affaddr{Dublin, Ohio 43017-6221}\\
%        \email{webmaster@marysville-ohio.com}
% % 3rd. author
% \alignauthor Lars Th{\o}rv{\"a}ld\titlenote{This author is the
% one who did all the really hard work.}\\
%        \affaddr{The Th{\o}rv{\"a}ld Group}\\
%        \affaddr{1 Th{\o}rv{\"a}ld Circle}\\
%        \affaddr{Hekla, Iceland}\\
%        \email{larst@affiliation.org}
% \and  % use '\and' if you need 'another row' of author names
% % 4th. author
% \alignauthor Lawrence P. Leipuner\\
%        \affaddr{Brookhaven Laboratories}\\
%        \affaddr{Brookhaven National Lab}\\
%        \affaddr{P.O. Box 5000}\\
%        \email{lleipuner@researchlabs.org}
% % 5th. author
% \alignauthor Sean Fogarty\\
%        \affaddr{NASA Ames Research Center}\\
%        \affaddr{Moffett Field}\\
%        \affaddr{California 94035}\\
%        \email{fogartys@amesres.org}
% % 6th. author
% \alignauthor Charles Palmer\\
%        \affaddr{Palmer Research Laboratories}\\
%        \affaddr{8600 Datapoint Drive}\\
%        \affaddr{San Antonio, Texas 78229}\\
%        \email{cpalmer@prl.com}
% }
% There's nothing stopping you putting the seventh, eighth, etc.
% author on the opening page (as the 'third row') but we ask,
% for aesthetic reasons that you place these 'additional authors'
% in the \additional authors block, viz.
% \additionalauthors{Additional authors: John Smith (The Th{\o}rv{\"a}ld Group, email: {\texttt{jsmith@affiliation.org}}) and Julius P.~Kumquat (The Kumquat Consortium, email: {\texttt{jpkumquat@consortium.net}}).} \date{30 July 1999}
% Just remember to make sure that the TOTAL number of authors
% is the number that will appear on the first page PLUS the
% number that will appear in the \additionalauthors section.

\maketitle
\begin{Introduction}
The intent of the original decentralised design of the Web was to make it easy for individuals to share and access information by simply setting up a web server and using a browser.  Yet, instead of individuals running their own servers, today, most people primarily use one of several large platform service providers.  Such platform  are continuing to grow at exponential rates, with some, such as Facebook, exceeding 1 billion users at the time of writing.  In exchange for this convenience, these platforms have assumed a critical place at the centres of people's information environments, creating not only an inextricable dependence on their services, but also the ability to manipulate individuals through targeted advertising, behavioural manipulation, and nearly constant surveillance.  Having now amassed a bulk of the world's personal information traces, such platforms now have the unprecedented capacity to further increase their dominance, through large scale statistical insights.  

Many have written previously about the potential dangers of such a centralised Web from a variety of perspectives.  % put some of those here. 
 
% As result,  such as in order to create increased dependence,  more effective behavioural exploitating and advertising.

In response, many efforts have sought to `re-decentralise' the web, that is allow end-users to rely less on such brokers, and retain greater control over their data.  In this paper, we critically interrogate what this means; is this a call to extend the current model of the Web or a call to fundamentally change how some aspect of how the Web currently works?

The precise path to achieve this vision of re-democratisation remains unclear, and much debated.  

 model of a Web client to provide end-users with information storage and serving capabilities?

We posit that, architecturally, with increasingly advanced capabilities of web browsers, the Web is ready for increased functionality centred at the end user for fostering greater responsiblity and control.  

 of democratisation remains unclear.  Implicit  Specifically, we In order for any such approaches to be successful, they must be simple, open and, essentially minimal.  

We first analyse 75 ``indie web'' projects that have this goal, deriving four categories of systems. Next, based on the capabilities such systems provide, we provide an analysis of the needs in greatest demand, which, beyond information storage and hosting, include universal addressibility in the face of dynamic network connectivity, private, secure and anonymous communications, and simple programmability.  We then derive a set of design requirements for extending Web clients, and provide the example of a decentralised microblogging environment that maintains the Web's principles of minimal  commitment to achieve maximum interoperability and support for future change.
\end{abstract}

\section{Introduction 2}
The increasing centralisation of the Web remains the greatest threats to its continued existence as a democratic and ubiquitous shared medium of communication. Although architected as a decentralised system to ensure fair and equal access to all users, today the Web has become a highly centralised environment, dominated by very large institutions that each control all of the traffic that flows within their respective borders.  The result is that such institutions harbour an disproportionately large percentage of Web traffic, and, in turn, exercise an unprecedented degree of control over its governance and operation.

The very definition of a democratic medium requires that its administration and governance lie with its individual participants.  In this paper, coinciding with the 25th anniversary of the Web, we look at current citizen-led approaches to ``take back the Web'' in order to understand how future Web architectures might support greater autonomy for Web end-users.  The goal of this analysis is to, first, identify what is seen to be most lacking about the current information environments on the Web, and second, to understand the variety of approaches taken to address such needs.  Indeed, merely the notion of ``taking it back'' is not necessarily clear; such approaches call variations of degrees under which the responsibility of being a data controller moves back to end-users and software under their control, running, in some cases directly within the browser itself.  It is for this purpose that survey both the ways and 
	
For the purpose of identifying functionality most needed towards enabling greater end-user autonomy of Web users, we start by identifying, collating, and the infastructure efforts recently produced by the ``indie web'' community.  From this analysis, we identify common functionality offered by each, and reflect on what such underlying needs such functionality  are meant to address.

In the second half of this paper, we select specific such functionality in the context of an example, focusing on microblogging service as an archetypal social platform.  In this analysis, we identify the basic data sharing activities such services currently provide to deliver their user experiences. For each of these activities, we consider adapting to a decentralised setting, and discuss the implications for web architectures in the future.

In the following sections, we first summarise the efforts of the \emph{indie web camp} movement towards building technologies for enabling end-users of the Web to take more control of their data. We then focus on this analysis of microblogging as a platform adapted to a decentralised setting.  Finally, we introduce two different approaches to decentralised Web architectures interoperating  through the Web in a decentralised manner to provide an effective microblogging platform that uses Web standards.

\section{Motivation}
Many have expressed a variety of views from the societal, cultural, to the psychological perils of the ever-widening gap in power and influence resulting from the consolidation of the world's information brokers to a few massive supersites, versus the rest of the Web, including individuals who consume and share information themeslves.

People have expressed many motivations for seeking to return to a more decentralised Web, and reduce reliance upon large online information brokers for various reasons, spanning many aspects of privacy \cite{}, the protection of free-speech \cite{}, and the desire to have greater control. Currently, Facebook ``controls'' an increasing amount of the media industry, spanning news to entertainment, by fully controlling how much attention and exposure each story gets.  A recent study surveydd that 40\% of American adults got their news from Facebook, the single largest source  This control has resulted in a number of peculiar behaviours 

\section{Survey of ``Indie Web'' efforts}

There has been a recent rise in efforts to 're-decentralise' the Web. However, as most of such efforts have themselves been highly distributed, a coherent picture of these efforts have not previously been assembled, in particular towards identifying the ways that such projects complement or overlap with one another. Our first goal, thus was to understand the space of indie web efforts to chronicle both the areas where progress has been made, and, second and perhaps more importantly, to identify the capabilities that developers participating in such efforts see as needed to empower individuals on the Web. 

In this section, we summarise the results of sampling and surveying a large number of such indie web projects, first to identify what each of these projects are meant to achieve, and second, to reflect upon the underlying needs that such projects indicate.

\subsection{Method}
We attempted to get a representative sample using a combination of approaches: first, we identified two large wikis directories dedicated to such projects, which included the \emph{Indie Web}\footnote{} and the \emph{Alternative Internet}\footnote{}. 

We then attempted to derive a set of categories out of the systems idetified.  For this, we used an iterative approach, in which we attempted to create the most specific descriptions that would characterise, at a high level, what each system accomplished while ensuring that each category had at least one system.

\subsection{Results: Systems Surveyed}

\begin{figure*}[tb]
\begin{tabular}{ l p{2.8in} p{2.3in} }
type & description & examples \\ \hline
  ``Personal Clouds'' & Largely standalone containers for hosting end-user web applications on personal hardware and VMs, including e-mail, calendaring tools, file storage and sharing & FreedomBox, Owncloud, arkOS, Sandstorm \\
  Anonymising Networks & support anonymous communciations, either at the packet-level or message-level, and robust/overlay routing & cjdns, i2p, bitmessage, briar \\
  Distributed Storage Environments & provide storage abstractions over multiple nodes & cjdns \\
  Other & other related services include: cryptocurrencies, identity providers, social network applications & Bitcoin, \\
\end{tabular}
\end{figure*}

From the YYY systems surveyed, we selected the XXX systems that were inherently decentralised and designed for user-

\subsubsection{Personal Cloud projects}

The first category consist of \emph{personal cloud} projects, which consist of software (and in some cases, hardware) platforms designed to allow end-users to more easily set-up, host and manage web applications for their personal or private use.  Such projects vary considerably in scope and complexity. Some projects encompass custom distributions of a popular FLOSS operating systems (typically GNU/Linux), designed to be hosted on low-cost embedded hardware, such as Raspberry PIs (e.g. xXX) or BeagleBones (e.g. XXX), or commodity hosting services (e.g. EC2 or Azure). 

On top of such underlying operating systems, typically lies an ensemble of services, typically consisting of mail servers, a standard Web server, file sharing services and a database. FreedomBox is one of the original and most comprehensive of such platforms, which offers both standard e-mail, web and file storage but also provides utilities for establishing secure communications (through PGP tools and tor).  Sandstorm\footnote{} takes a slightly different approach, designing a docker-inspired modular architecture for installing and configuring web-apps that have traditionally required manual web-server configuration. 

\subsubsection{Encrypted/Anonymous Overlay Networks}

The second category consist of tools for secure and anonymous communications.  These projects range from network-level systems that provide addressing and routing (such as cjdns and i2p) to end-user applications (such as Bitmessage Briar), to projects that straddle the two (such as tor). While some projects work to route existing IP- and TCP- applications (including the Web), others are application specific (e.g. specific to messaging, despite being payload independent).

As described later, the projects in this category seem to all have several features in common , they provide different degrees of support for the underlying connectivity;including addressing and message routing) , while others address connectivity issues, including partially disconnection and firewall traversal

% cjdns
% tor
% i2p
% Bitmessage
% Briar

\subsection{Distributed Storage Environments}
% Bitcloud 
% DHTs
% Bittorrent 
% MaidSafe
% BaseParadigm
% Avatar

% \subsection{Results: Needs}

% \subsubsection{Addressability}
% \subsubsection{Reachability}
% \subsubsection{Anonymity}
% \subsubsection{Secure \& authenticated communication}
% \subsubsection{Pub/sub functionality}
% \subsubsection{Data storage}
% \subsubsection{Distributed Consensus \& Fault Tolerance}
% 	Askemos
% \subsubsection{Application Development Environments}
% \subsection{End-user service composition}

% \section{Desired Properties}

% Working with the existing Web 
% Decentralised and Disconnectable
% 	No central locus of control
% “Centralised Experience”
% Efficiency
% End-user Control
% Privacy and Anonymity
% Technical Implications

% \subsection{Implementations}
% Case Example: Microblogging
% 	Broadcast posts/Timeline Syndication
% 	Replies and comments
% Retweets/Reblogs
% Mentions
% Search, Trending	
% Anonymous Posting
% Evaluation
% Implications and Future Work

% \section{Conclusion}
% The Web is by no means a purely technical system; socio-economic factors have largely been responsible for its centralisation.  However, we believe that architectural considerations and patterns can reduce the barriers needed for the developer community to

%ACKNOWLEDGMENTS are optional
% \section{Acknowledgments}
\bibliographystyle{abbrv}
% \bibliography{www2015-redecentalise}  % sigproc.bib is the name of the Bibliography in this case
\balancecolumns
% That's all folks!
\end{document}
